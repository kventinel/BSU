\documentclass[10pt]{scrartcl}
\usepackage[utf8]{inputenc}
\usepackage[english,russian]{babel}
\usepackage{indentfirst}
\usepackage{misccorr}
\usepackage{graphicx}
\usepackage{verbatim}
\usepackage[fleqn]{amsmath}
\usepackage{tikz}
\usepackage{diagbox}
\usepackage[a4paper,margin=1.0in,footskip=0.25in]{geometry}
\makeatletter
\newcommand{\verbatimfont}[1]{\renewcommand{\verbatim@font}{\ttfamily#1}}
\begin{document}
\author{Рак Алексей}
\title{Optimization_Methods/Homework1/Rak}
\begin{titlepage}
		\centering
		{\scshape\LARGE Белорусский Государственный Университет \par}
        \vfill
        {\scshape\LARGE Методы Численного Анализа\par}
        \vspace{1cm}
        {\scshape\LARGE Лабораторная работа 4\par}
        \vspace{1cm}
        {\scshape\LARGE Интерполирование с помощью многочлена Чебышева\par}
        \vspace{2cm}
        {\LARGE Рак Алексей\par}
        \vfill
        {\large \today}
\end{titlepage}
\section*{Постановка Задачи}\noindent
Минимизировать остаток интерполирования, используя многочлен Чебышева. Для этого с его помощью выберем узлы
интерполирования на нужном отрезке $[1, 2]$ и проведём по ним интерполяцию многочленом Лагранжа. Вычислим его 
значения в контрольных точках $x^*, x^{**}, x^{***}$. Оценим результат и сравним его с предыдущим.
\section*{Данные}\noindent
Функция: $f(x) = 1.3 * e^x - 0.3 * sin(x)$\\
$x^{*} = \frac{31}{30}, \ x^{**} = \frac{46}{30}, \ x^{***} = \frac{59}{30}$
\section*{Алгоритм решения и формулы}\noindent
Идея метода состоит в выборе таких узлов для построения интерполяционного полинома, чтобы в оценке:
\begin{gather*}
\Vert f(x) - P_n(x) \Vert \leq \frac{\Vert f^{(n + 1)}(x)w(x)\Vert}{(n+1)!}
\end{gather*}
многочлен $w(x)$ был наименее отклоняющимся от нуля, что достигается на Чебышёвском наборе узлов.
\begin{gather*}
x_m = \frac{a + b}{2} + \frac{b - a}{2}cos\left(\frac{\pi}{n + 1}\left(m + \frac{1}{2}\right)\right)
\ \ \ \ \ \ \ \ \ \ m = \overline{0, n}
\end{gather*}
В таком случае оценка погрешности принимает вид:
\begin{gather*}
\Vert f(x) - P_n(x)\Vert \leq \frac{\Vert f^{(n + 1)}(x)\Vert(b - a)^{n + 1}2^{1 - 2(n + 1)}}{(n + 1)!}
\end{gather*}
\section*{Листинг}
\verbatimfont{\small}
\begin{verbatim}
#include <cmath>
#include <iostream>
#include <vector>

const int n = 11;
const double a = 1.3;
const double x1 = 31.0 / 30;
const double x2 = 46.0 / 30;
const double x3 = 59.0 / 30;
std::vector<double> nodes(n);
std::vector<double> fx(n);
const double right = 2;
const double left = 1;

double f(double x) {
    return a * exp(x) + (1 - a) * sin(x);
}

double f1(double x) {
    return a * exp(x) + (1 - a) * cos(x);
}

double f2(double x) {
    return a * exp(x) - (1 - a) * sin(x);
}

double f3(double x) {
    return a * exp(x) - (1 - a) * cos(x);
}

int factorial(int x) {
    if (x == 1 || x == 0)
        return 1;
    return x * factorial(x - 1);
}

double Lagranj(double x) {
    double ans = 0;
    for (int i = 0; i < n; i++) {
        double term = 1, xi = nodes[i];
        for (int j = 0; j < n; j++) {
            if (i == j) continue;
            double xj = nodes[j];
            term = term * (x - xj) / (xi - xj);
        }
        term *= f(xi);
        ans += term;
    }
    return ans;
}

void createNodes() {
    double add = (right + left) / 2., mul = (right - left) / 2.;
    for (int i = 0; i < n; ++i) {
        nodes[i] = add + mul * cos(M_PI * (2 * (n - i) + 1) / 2 / (n + 1));
        fx[i] = f(nodes[i]);
    }
}

double calcR_true(double x) {
    return fabs(Lagranj(x) - f(x));
}

double findTheoreticalResidual() {
    double mul = (right - left) / 2.;
    for (int i = 1; i < n; ++i)
        mul *= ((right - left) / (i + 1) / 4.);
    switch (n % 4) {
        case 0:
            return f1(right) * mul;
        case 1:
            return f2(right) * mul;
        case 2:
            return f3(right) * mul;
        case 3:
            return f(right) * mul;
    }
    return 0;
}

int main() {
    createNodes();
    std::cout << "nodes: " << std::endl;
    for (int i = 0; i < n; i++) {
        std::cout << nodes[i] << " ";
    }
    std::cout << std::endl;
    std::cout << "expected: " << findTheoreticalResidual() << std::endl;
    std::cout << "x*:" << std::endl;
    std::cout << "resalt: " << Lagranj(x1) << std::endl;
    std::cout << "true: " << calcR_true(x1) << std::endl;
    std::cout << "x**:" << std::endl;
    std::cout << "resalt: " << Lagranj(x2) << std::endl;
    std::cout << "true: " << calcR_true(x2) << std::endl;
    std::cout << "x***:" << std::endl;
    std::cout << "resalt: " << Lagranj(x3) << std::endl;
    std::cout << "true: " << calcR_true(x3) << std::endl;
    return 0;
}
\end{verbatim}
\section*{Результаты и вывод}\noindent
Выходные данные:\\
Ноды:\\ 
$1.00428 1.03806 1.10332 1.19562 1.30866 1.43474 1.56526 1.69134 1.80438 1.89668 1.96194$ \\
Ожидаемая погрешность: $1.1149e^{-13}$\\
Точка $x^{*}$:\\
Результат: $3.39584$\\
Истинная погрешность: $4.44089e^{-15}$\\
Точка $x^{**}$:\\
Результат: $5.72389$\\
Истинная погрешность: $2.4869e^{-14}$\\
Точка $x^{***}$:\\
Результат: $9.01406$\\
Истинная погрешность: $1.84741e^{-13}$\\
Вывод:\\
Данный метод является одним из самых точных методов из числа рассмотренных, в силу того, что в этом методе происходит выбор самих узлов. Выбранные узлы являются корнями многочлена Чебышева, а многочлен Чебышева есть наиболее близкая к нулю функция, то есть мы приближаем остаток интерполирования к нулю наилучшим образом.\\
Во всех точках погрешность стала меньше, это наиболее заметно в первой точке, чуть менее заметно во второй точке
и практически не заметно в третей. Это можно списать на то, что вначале рассматриваемого отрезка 
"концентрация узлов" вышла более "плотной", поэтому точность понижается при приближении к третьей точке.\\
\end{document}