\documentclass[10pt]{scrartcl}
\usepackage[utf8]{inputenc}
\usepackage[english,russian]{babel}
\usepackage{indentfirst}
\usepackage{misccorr}
\usepackage{graphicx}
\usepackage{verbatim}
\usepackage[fleqn]{amsmath}
\usepackage{tikz}
\usepackage{diagbox}
\usepackage[a4paper,margin=1.0in,footskip=0.25in]{geometry}
\makeatletter
\newcommand{\verbatimfont}[1]{\renewcommand{\verbatim@font}{\ttfamily#1}}
\begin{document}
\author{Рак Алексей}
\title{Optimization_Methods/Homework1/Rak}
\begin{titlepage}
		\centering
		{\scshape\LARGE Белорусский Государственный Университет \par}
        \vfill
        {\scshape\LARGE Методы Численного Анализа\par}
        \vspace{1cm}
        {\scshape\LARGE Лабораторная работа 6\par}
        \vspace{1cm}
        {\scshape\LARGE Интерполирование сплайном третьего порядка\par}
        \vspace{2cm}
        {\LARGE Рак Алексей\par}
        \vfill
        {\large \today}
\end{titlepage}
\section*{Постановка Задачи}\noindent
По заданной равномерной сетке узлов $x_i \in [1, 2], x_i = 1 + ih, h = \frac{2 - 1}{n}, i = \overline{0, n},
n = 10$ для функции $f(x) = 1.3e^x - 0.3\sin x$ требуется построить интерполяционный кубический сплайн.
Рассчитать истинную погрешность в точках:
\begin{gather*}
x^* = x_0 + \frac{h}{3}, \ x^{**} = x_5 + \frac{h}{3}, \ x^{***} = x_{10} - \frac{h}{3}\\
\end{gather*}
\section*{Алгоритм решения и формулы}\noindent
Строим систему следующего вида для вычитания моментов $M_i = S''(x_i)$
\begin{gather*}
\begin{cases}
2M_0 - \lambda_0 M_1 = d_0, i = 0\\
-\nu M_{i - 1} + 2M_i - \lambda_i M_{i + 1} = d_i, i =\overline{1, n - 1}\\
-\nu_n M_{n - 1} + 2M_n = d_n, i = n
\end{cases}
\end{gather*}
2 дополнительных условия зададим в виде:\\
\begin{gather*}
\begin{cases}
d_0 = 2f''(x_0)\\
d_n = 2f''(x_n)
\end{cases}
\end{gather*}
откуда
\begin{gather*}
\begin{cases}
\lambda_0 = 0\\
\nu_n = 0
\end{cases}
\end{gather*}
Остальные коэффициенты находим по формулам:
\begin{gather*}
\lambda_i = - \frac{h_i+1}{h_i+h_{i+1}} = -0.5, i = \overline{1, n - 1}\\
\nu_i = - \frac{h_i}{h_i + h_{i + 1}} = -0.5, i = \overline{1, n - 1}\\
d_i = \frac{3}{h}\left(\frac{f_{i + 1} - f_i}{h} - \frac{f_i - f{i - 1}}{h}\right), i = \overline{1, n - 1}\\
\end{gather*}
Решим данную систему методом левой прогонки и найдём моменты $M_i$.\\
Интерполяционный сплайн на отрезке $[x_{i-1}, x_i], i = \overline{1, n}$ строим по формуле:\\
$S_{i - 1}(x) = M_{i - 1}\frac{(x_i - x)^3}{6h} + M_i\frac{(x - x_{i - 1})^3}{6h} + \frac{x_i - x}{h}
\left(f_{i - 1} - \frac{M_{i - 1}h^2}{6}\right) + \frac{x - x_{i - 1}}{h}
\left(f_i - \frac{M_ih^2}{6}\right)$\\
Для точек $x^*, x^{**}, x^{***}$ считаем истинную погрешность:\\
$R_{true}(x) = |f(x) - s_k(x)|$, где $x_k \leq x \leq x_{k + 1}$
\section*{Листинг}
\verbatimfont{\small}
\begin{verbatim}
#include <cmath>
#include <iomanip>
#include <iostream>

class Spline {
public:
    const double coef = 1.3;
    const int N = 10;
    double A, B, h;
    double *x, *f, *l, *v, *d, *c;

    Spline() {
        A = 1, B = 2, h = 0.1;
        x = new double[N + 1];
        f = new double[N + 1];
        l = new double[N];
        v = new double[N];
        d = new double[N + 1];
        c = new double[N + 1];
    }

    void initSweepCoefs() {
        for (int i = 0; i < N + 1; i++) {
            x[i] = A + i * h;
            f[i] = coef * exp(Spline::x[i]) + (1 - coef) * sin(Spline::x[i]);
        }
        l[0] = 0;
        v[N - 1] = 0;
        c[0] = c[N] = 2;
        d[0] = 2 * (coef * exp(A) - (1 - coef) * sin(A));
        d[N] = 2 * (2.1 * exp(B) - (1 - coef) * sin(B));
        for (int i = 0; i < N - 1; i++) {
            l[i + 1] = v[i] = -0.5;
            c[i + 1] = 2;
            d[i + 1] = (3 / h) * (((f[i + 2] - f[i + 1]) / h) - ((f[i + 1] - f[i]) / h));
        }
    }

    double S(int i, double x, double *mom) {
        return (mom[i - 1] * pow(Spline::x[i] - x, 3) / 6 * h) +
               (mom[i] * pow(x - Spline::x[i - 1], 3) / 6 * h) +
               ((Spline::x[i] - x) / h) * (f[i - 1] - ((mom[i - 1] * pow(h, 2)) / 6)) +
               ((x - Spline::x[i - 1]) / h) * (f[i] - ((mom[i] * pow(h, 2)) / 6));
    }

    double *solveMatrix() {
        double *x = new double[N + 1];
        double m;
        for (int i = 1; i < N + 1; i++) {
            m = v[i] / c[i - 1];
            c[i] = c[i] - m * l[i - 1];
            d[i] = d[i] - m * d[i - 1];
        }
        x[N] = d[N] / c[N];
        for (int i = N - 1; i >= 0; i--)
            x[i] = (d[i] - l[i] * x[i + 1]) / c[i];
        return x;
    }
};

int main() {
    Spline *spline = new Spline();
    spline->initSweepCoefs();
    double *moments = spline->solveMatrix();
    double *xch = new double[3];
    double *fch = new double[3];
    double *Rtrue = new double[3];
    double *Sch = new double[3];
    xch[0] = spline->x[0] + (spline->h / 3);
    xch[1] = spline->x[5] + (spline->h / 3);
    xch[2] = spline->x[10] - (spline->h / 3);
    int ind[3] = {1, 6, 10};
    for (int i = 0; i < 3; i++) {
        fch[i] = spline->coef * exp(xch[i]) + (1 - spline->coef) * sin(xch[i]);
        Sch[i] = spline->S(ind[i], xch[i], moments);
        Rtrue[i] = fabs(fch[i] - Sch[i]);
        std::cout << "f[" << i << "]: " << fch[i] << " " << "S[" << i << "]: " << Sch[i] << " "
                  << "Pogr[" << i << "]: " << Rtrue[i] << std::endl;
    }
    return 0;
}
\end{verbatim}
\section*{Результаты и вывод}\noindent
Выходные данные:\\
f[0]: 3.39584 S[0]: 3.3902 Pogr[0]: 0.00564207\\
f[1]: 5.72389 S[1]: 5.69945 Pogr[1]: 0.0244403\\
f[2]: 9.01406 S[2]: 8.99745 Pogr[2]: 0.0166138\\
Вывод:\\
Исходя из полученных результатов, можно заметить, что погрешность вычислений имеет большую величину. Это объясняется тем, что на каждом отрезке строится полином третьей степени (например, в других работах строились полиномы больших степеней).\\
Преимущество данного подхода заключается в том, что строится несколько полиномов для всех маленьких отрезков, а не один для всего отрезка интерполирования. Этот метод намного лучше отражает поведение функции на каждом из 
маленьких отрезков.
\end{document}