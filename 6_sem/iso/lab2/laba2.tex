\documentclass[a4paper,14pt]{extreport}

\usepackage[T1,T2A]{fontenc}
\usepackage[utf8]{inputenc}
\usepackage[figure,table]{totalcount}
\usepackage{lastpage}
\usepackage{graphics}
\graphicspath{ {images/} }
\usepackage{subcaption}
\usepackage{paralist}
\usepackage{amsthm, amsmath, amsfonts, amssymb}
\usepackage{mathtools} % \mathclap
\usepackage{bm}
\usepackage{dsfont}
\usepackage{hyperref}
\usepackage{tabularx}
\usepackage{graphicx}
\usepackage{multirow}
\usepackage{comment}
\usepackage{xcolor, colortbl}
\usepackage{xifthen, xspace}

%%\usepackage{styles/bsumain}
\usepackage{blindtext}
\usepackage{hyperref}

\newcommand*{\jobtitle}[1]{\gdef\@jobtitle{#1}}
\newcommand{\quotes}[1]{``#1''}

\jobtitle{ИССЛЕДОВАНИЕ ОПЕРАЦИЙ}

\begin{document}

%\input{titlepage.tex}
\begin{titlepage}
		\centering
		{\scshape\Large БГУ \par}
        \vfill
        {\scshape\LARGE Исследование операций\par}
        \vspace{2cm}
        {\LARGE Рак Алексей\par}
        \vfill
        {\large \today}
\end{titlepage}
\setcounter{page}{2}

\begin{center}
	\textbf{1.2. Многокритериальные задачи}
\end{center}

\textbf{Задача 20}

Имеется вычислительная сеть, имеющая топологию “звезда”, то есть множество $N=\{1, 2, ..., n\}$ параллельно работающих компьютеров, первый из которых выполняет роль концентратора. Имеется задание объёмом $W$ единиц информации, которое необходимо выполнить на данных компьютерах. При этом концентратор помимо выполнения своей части задания, которая составляет не более 25\% объёма задания, производит также обмен информацией между компьютерами. Производительность $i$-го компьютера составляет $c_i$ единиц объёма информации в единицу времени по выполнению задания, формированию пересылаемой информации и обработке полученной информации. Между компьютерами должен происходить обмен информацией по каналам связи, через концентратор, но с условием сохранения баланса, то есть суммарный объём информации, передаваемой от компьютера другим компьютерам, должен быть равен суммарному объёму информации, получаемому компьютером. Оба суммарных объёма составляют 30\% объёма информации задания, первоначально переданного компьютеру. Для концентратора объём обмена с компьютерами состоит из 30\% процентов информации своей части задания и всей информации передаваемой между компьютерами. Скорость передачи по каналу $(1, j)$ равна $v_{1j}$ единиц информации в единицу времени. Пусть также задано множество $M \subseteq \{(1,j) \mid j \in N, j \ne 1\}$ “контролируемых” каналов связи. Требуется распределить задание между компьютерами, минимизирующее время выполнения задания и минимизирующее суммарный трафик по “контролируемым” каналам связи.
\newline

\textbf{Модель задачи.} 

Управляемые переменные: $x_i$ -- объем задания для $i$-го компьютера, $i = \overline{1, n}$.

Неуправляемые переменные: $y_{ij}$ -- объем информации, передаваемой от $i$-го компьютера к $j$-ому, $i, j = \overline{1, n}$.
\newline

Ограничения:
$$
	x_i \ge 0, i =\overline{1, n},
$$ $$
	x_1 \le 0.25 W,
$$ $$
	\sum_{i=1}^n x_i = W,
$$ $$
	\sum_{j=1}^n y_{ij} = \sum_{j=1}^n y_{ji} = 0.3x_i, j \ne i, i = \overline{2, n},
$$ $$
	\sum_{j=2}^n y_{1j} = \sum_{j=2}^n y_{j1} = 0.3 x_1 + \sum_{i=2}^n \sum_{j=2}^n y_{ij}.
$$

Целевые функции:
$$
	f_1(x) = \max_{i=\overline{1,n}} \frac{x_i}{c_i} \to \min
$$ $$
	f_2(x) = \sum_{(1, i) \in M} 0.6x_i \to \min
$$

\textbf{Задача 21}

Задано $n$ работ, подлежащих выполнению. Для выполнения работ можно привлечь $m \le n$ исполнителей. Время выполнения$i$-ой работы $j$-ым исполнителем равно $t_{ij} > 0$, затраты на выполнение $i$-ой работы $j$-ым исполнителем соответственно $c_{ij} > 0$. Каждая работа может выполняться только одним исполнителем, соответственно каждый работник должен быть назначен хотя бы на выполнение одной работы.

Требуется найти назначение исполнителей на работы которое:
\begin{itemize}
\item[а)] минимизирует общие затраты и время выполнения работ;
\item[б)] минимизирует общие затраты и общее время выполнения работ;
\item[в)] минимизирует общие затраты при заданном нормативном времени $T$ выполнения всех работ.
\end{itemize}

\textbf{Модель задачи.} 

Управляемые переменные: $t_{ij}$ -- время выполнения$i$-ой работы $j$-ым исполнителем, $i = \overline{1, n}$, $j = \overline{1, m}$, $c_{ij}$ -- затраты на выполнение $i$-ой работы $j$-ым исполнителем, $i = \overline{1, n}$, $j = \overline{1, m}$.

Неуправляемые переменные: 
\newline
$x_{ij} = \begin{cases}
	1, & \text{если j-ый работник выполняет i-ую работу,}\\
	0, & \text{иначе.}\\
\end{cases}$
\newline

Ограничения:
$$
	\sum_{i=1}^n x_{ij} \ge 1, \quad \sum_{j=1}^m x_{ij} = 1,
$$
т. е. каждый работник может быть назначен на 1 и более работ, и на каждую работу может быть назначен ровно 1 работник.
\newline

Целевые функции:
\begin{itemize}
	\item[а)] общие затраты и время выполнения работ:
$$
	f_0 = \sum_{i=1}^n \sum_{j=1}^m c_{ij} x_{ij} \to \min,
$$ $$
	f_i = \sum_{j=1}^m t_{ij} x_{ij} \to \min, \quad \forall i=\overline{1, n};
$$

	\item[б)] общие затраты и общее время выполнения работ:
$$
	f_0 = \sum_{i=1}^n \sum_{j=1}^m c_{ij} x_{ij} \to \min,
$$ $$
	f_1 = \sum_{i=1}^n \sum_{j=1}^m t_{ij} x_{ij} \to \min;
$$

	\item[в)] введём дополнительное ограничение на время выполнения всех работ:
$$
	\sum_{i=1}^n \sum_{j=1}^m t_{ij} x_{ij} = T,
$$
тогда целевая функция имеет вид:
$$
	f_0 = \sum_{i=1}^n \sum_{j=1}^m c_{ij}x_{ij} \to \min.
$$
\end{itemize}

\begin{center}
	\textbf{1.3. Сужение неопределенности. Компромиссы Парето}
\end{center}

\textbf{Задача 22}

Найти множество Парето в следующих многокритериальных задачах:
\begin{itemize}
	\item[a)] $f_1(x) \to \max$, $f_2(x) \to \max$, где $f_1(x) = ax+b(a-x)$, $f_2(x) = x^{\alpha}(1-x)^{\beta}$, при условии $0 \le x \le 1$. Здесь $a, b, \alpha, \beta$ -- положительные константы;

	\item[b)] $f_1(x_1, x_2) \to \max$, $f_2(x_1, x_2) \to \max$, где $f_1(x_1, x_2) = x_1 + x_2$, $f_2(x_1, x_2) = x_1^2 - x_2^2$, при условии $0 \le x_i \le 1$, $i = 1,2$;

	\item[c)] $f_i(x_1, x_2, x_3) \to \max$, $i=\overline{1, 3}$, где $f_1(x_1, x_2, x_3) = \min\{x_2, x_3\}$, $f_2(x_1, x_2, x_3) = \min\{x_1, x_3\}$, $f_3(x_1, x_2, x_3) = \min\{x_1, x_2\}$, при условии $x_1 + x_2 + x_3 = 1, x_i \ge 0, i = 1,2,3$.
\end{itemize}

\textbf{Решение.} 

\begin{itemize}
	\item[a)] $f_1(x)$ -- линейная функция, возрастает, если $a < b$ , и убывает, если $a > b$. Вторая функция $f_2$ достигает максимума в точке $\frac{\alpha}{\alpha+\beta}$. Таким образом,
	\begin{itemize}
		\item если $a=b$, то множество Парето имеет вид $\{\frac{\alpha}{\alpha+\beta}\}$;
		\item если $a<b$, то множество Парето имеет вид $\left[ 0, \frac{\alpha}{\alpha+\beta}\right]$;
		\item если $a>b$, то множество Парето имеет вид $\left[\frac{\alpha}{\alpha+\beta}, 1\right]$.
	\end{itemize}

	\item[b)] Первая функция возрастает с ростом $x_1$ или $x_2$, вторая – возрастает с ростом $x_1$, убывает с ростом $x_2$. Значит, множество Парето имеет вид $\{1\} \times [0, 1]$, так как при фиксированном $x_2$ обе функции достигают максимума в точке $x_1 = 1$ и возрастают с ростом $x_1$.

	\item[c)] Выразим $x_3 = 1 - x_1 - x_2$. Тогда получаем следующее множество Парето:
\begin{equation*}
\begin{split}
	\{ (x_1, x_2, x_3) \mid x_1 = x_2 \ge \frac{1}{3} \} \cup \{ (x_1, x_2, x_3) \mid x_1 = x_3 \ge \frac{1}{3} \} \cup \\
\cup \{ (x_1, x_2, x_3) \mid x_2 = x_3 \ge \frac{1}{3} \}
\end{split}
\end{equation*}

\end{itemize}

\begin{center}
	\textbf{2.1. Матричные игры. Разрешимость в чистых стратегиях}
\end{center}

\textbf{Задача 1}

\begin{itemize}

\item[a)] Показать, что матричная игра с матрицей $H=(h_{ij})_{n \times m}$  имеет решение в чистых стратегиях, и найти такое решение, если $h_{ij} = f(i) - g(j)$.

\textbf{Решение.} 

Найдём нижнее и верхнее значения игры:
$$
\underline{I} = \max_i \min_j \left( f(i) - g(j) \right) = \max_i \left( f(i) - \max_j g(j) \right) = \max_i f(i) - \max_j g(j)
$$ $$
\overline{I} = \min_j \max_i \left( f(i) - g(j) \right) = \min_j \left( \max_i f(i) - g(j) \right) = \max_i f(i) - \max_j g(j)
$$ $$
\underline{I} = \overline{I} = \max_i f(i) - \max_j g(j)
$$

Значит, игра разрешима в чистых стратегиях.
Пара стратегий $(i_0, j_0)$ таких, что $\max_i f(i) = f(i_0)$, $\max_j g(j) = g(j_0)$, -- решение игры.
\newline

\item[b)] Показать, что матричная игра с матрицей $H=(h_{ij})_{n \times m}$  имеет решение в чистых стратегиях, и найти такое решение, если $h_{ij} = f(i) + g(j)$.

\textbf{Решение.} 

Найдём нижнее и верхнее значения игры:
$$
\underline{I} = \max_i \min_j \left( f(i) + g(j) \right) = \max_i \left( f(i) + \min_j g(j) \right) = \max_i f(i) + \min_j g(j)
$$ $$
\overline{I} = \min_j \max_i \left( f(i) + g(j) \right) = \min_j \left( \max_i f(i) + g(j) \right) = \max_i f(i) + \min_j g(j)
$$ $$
\underline{I} = \overline{I} = \max_i f(i) + \min_j g(j)
$$

Значит, игра разрешима в чистых стратегиях.
Пара стратегий $(i_0, j_0)$ таких, что $\max_i f(i) = f(i_0)$, $\min_j g(j) = g(j_0)$, -- решение игры.
\newline

\item[c)] Показать, что матричная игра с матрицей $H=(h_{ij})_{n \times m}$  имеет решение в чистых стратегиях, и найти такое решение, если

\[
H = \begin{bmatrix}
	a & b\\
	c & d\\
	a & d\\
	c & b\\
\end{bmatrix}
\]

\textbf{Решение.} 

\item[f)] Показать, что матричная игра с матрицей $H=(h_{ij})_{n \times m}$  имеет решение в чистых стратегиях, и найти такое решение, если $n = m$ и для любых $i, j, k, 1 \le i, j, k \le m$, имеет место тождество $h_{ij}+h_{jk} + h_{ki}= 0$.

\textbf{Решение.} 

Найдём нижнее и верхнее значения игры с учётом $h_{ij} = -h_{jk} - h_{ki}$, $\forall k$:
$$
\underline{I} = \max_i \min_j h_{ij} = \max_i \min_j \left( -h_{jk} - h_{ki} \right) = -\max_j h_{jk} - \min_i h_{ki}
$$ $$
\overline{I} = \min_j \max_i h_{ij} = \min_j \max_i \left( -h_{jk} - h_{ki} \right) = -\max_j h_{jk} - \min_i h_{ki}
$$ $$
\underline{I} = \overline{I} = -\max_j h_{jk} - \min_i h_{ki}
$$

Значит, игра разрешима в чистых стратегиях.
\newline

\end{itemize}

\begin{center}
	\textbf{2.2. Матричные игры. Смешанные стратегии. Сведение к задаче линейного программирования}
\end{center}

\par\noindent{\bf Задача~5.} \vskip 6pt
Каждый из игроков имеет 10 чистых стратегий: поставить все 3 фишки на одну из трёх позиций (пронумеруем эти стратегии от 1 до 3), поставить 2 фишки на одну из трёх позиций и ещё одну на оставшиеся две (пронумеруем эти стратегии от 4 до 9; стратегия 4 - поставить 2 фишки на первую позицию и ещё одну на вторую, стратегия 5 - поставить 2 фишки на первую позицию и ещё одну на третью, стратегия 6 - поставить 2 фишки на вторую позицию и ещё одну на первую и так далее), поставить все 3 фишки на разные позиции (дадим этой стратегии номер 10). \\Составим матрицу выигрышей:
\begin{center}
$H(i, j) =
 \begin{pmatrix}
  0 & 3 & 3 & 1 & 1 & 2 & 3 & 2 & 3 & 2 \\
  3 & 0 & 3 & 2 & 3 & 1 & 1 & 3 & 2 & 2 \\
  3 & 3 & 0 & 3 & 2 & 3 & 2 & 1 & 1 & 2 \\
  1 & 2 & 3 & 0 & 1 & 1 & 2 & 2 & 2 & 1 \\
  1 & 3 & 2 & 1 & 0 & 2 & 2 & 1 & 2 & 1 \\
  2 & 1 & 3 & 1 & 2 & 0 & 1 & 2 & 2 & 1 \\
  3 & 1 & 2 & 2 & 2 & 1 & 0 & 2 & 1 & 1 \\
  2 & 3 & 1 & 2 & 1 & 2 & 2 & 0 & 1 & 1 \\
  3 & 2 & 1 & 2 & 2 & 2 & 1 & 1 & 0 & 1 \\
  2 & 2 & 2 & 1 & 1 & 1 & 1 & 1 & 1 & 0
 \end{pmatrix}$
\end{center}
$\displaystyle\underline{I} = \max_{i}\min_{j}H(i, j) = 0$ \\
$\displaystyle\overline{I} = \min_{j}\max_{i}H(i, j) = 2 \neq \underline{I}$ \\
Значит, игра неразрешима в чистых стратегиях. Построим пару двойственных задач линейного программирования, предварительно добавив 1 ко всем элементам матрицы, чтобы они стали положительными: \\
$\begin{cases}
\displaystyle\sum_{i=1}^{10}x_i \to \min \\
x_1 + 4x_2 + 4x_3 + 2x_4 + 2x_5 + 3x_6 + 4x_7 + 3x_8 + 4x_9 + 3x_{10} \geq 1 \\
4x_1 + x_2 + 4x_3 + 4x_4 + 3x_5 + 4x_6 + 2x_7 + 2x_8 + 3x_9 + 3x_{10} \geq 1 \\
4x_1 + 4x_2 + x_3 + 4x_4 + 3x_5 + 4x_6 + 3x_7 + 2x_8 + 2x_9 + 3x_{10} \geq 1 \\
2x_1 + 3x_2 + 4x_3 + x_4 + 2x_5 + 2x_6 + 3x_7 + 3x_8 + 3x_9 + 2x_{10} \geq 1 \\
2x_1 + 4x_2 + 3x_3 + 2x_4 + x_5 + 3x_6 + 3x_7 + 2x_8 + 3x_9 + 2x_{10} \geq 1 \\
3x_1 + 2x_2 + 4x_3 + 2x_4 + 3x_5 + x_6 + 2x_7 + 3x_8 + 3x_9 + 2x_{10} \geq 1 \\
4x_1 + 2x_2 + 3x_3 + 3x_4 + 3x_5 + 2x_6 + x_7 + 3x_8 + 2x_9 + 2x_{10} \geq 1 \\
3x_1 + 4x_2 + 2x_3 + 3x_4 + 2x_5 + 3x_6 + 3x_7 + x_8 + 2x_9 + 2x_{10} \geq 1 \\
4x_1 + 3x_2 + 2x_3 + 3x_4 + 3x_5 + 3x_6 + 2x_7 + 2x_8 + x_9 + 2x_{10} \geq 1 \\
3x_1 + 3x_2 + 3x_3 + 2x_4 + 2x_5 + 2x_6 + 2x_7 + 2x_8 + 2x_9 + x_{10} \geq 1 \\
x_i \geq 0\; i=\overline{1, 10}
\end{cases}$\\
$\begin{cases}
\displaystyle\sum_{i=1}^{10}y_i \to \max \\
y_1 + 4y_2 + 4y_3 + 2y_4 + 2y_5 + 3y_6 + 4y_7 + 3y_8 + 4y_9 + 3y_{10} \leq 1 \\
4y_1 + y_2 + 4y_3 + 4y_4 + 3y_5 + 4y_6 + 2y_7 + 2y_8 + 3y_9 + 3y_{10} \leq 1 \\
4y_1 + 4y_2 + y_3 + 4y_4 + 3y_5 + 4y_6 + 3y_7 + 2y_8 + 2y_9 + 3y_{10} \leq 1 \\
2y_1 + 3y_2 + 4y_3 + y_4 + 2y_5 + 2y_6 + 3y_7 + 3y_8 + 3y_9 + 2y_{10} \leq 1 \\
2y_1 + 4y_2 + 3y_3 + 2y_4 + y_5 + 3y_6 + 3y_7 + 2y_8 + 3y_9 + 2y_{10} \leq 1 \\
3y_1 + 2y_2 + 4y_3 + 2y_4 + 3y_5 + y_6 + 2y_7 + 3y_8 + 3y_9 + 2y_{10} \leq 1 \\
4y_1 + 2y_2 + 3y_3 + 3y_4 + 3y_5 + 2y_6 + y_7 + 3y_8 + 2y_9 + 2y_{10} \leq 1 \\
3y_1 + 4y_2 + 2y_3 + 3y_4 + 2y_5 + 3y_6 + 3y_7 + y_8 + 2y_9 + 2y_{10} \leq 1 \\
4y_1 + 3y_2 + 2y_3 + 3y_4 + 3y_5 + 3y_6 + 2y_7 + 2y_8 + y_9 + 2y_{10} \leq 1 \\
3y_1 + 3y_2 + 3y_3 + 2y_4 + 2y_5 + 2y_6 + 2y_7 + 2y_8 + 2y_9 + y_{10} \leq 1 \\
y_i \geq 0\; i=\overline{1, 10}
\end{cases}$\\
Решив полученную пару задач (например, симплекс-методом), найдём оптимальные решения:\\
$x_1 = x_2 = x_3 = \frac{1}{9}$ \\
$x_i = 0,\; i=\overline{4, 10}$ \\
$y_1 = y_2 = y_3 = \frac{1}{9}$ \\
$y_i = 0\,; i=\overline{4, 10}$ \\
Значение игры: \\
$\displaystyle I = \frac{1}{\sum_{i=1}^{10}x_i} - 1 = 3 - 1 = 2$
Оптимальные смешанные стратегии: \\
$\displaystyle p_1 = p_2 = p_3 = \frac{1}{9}\cdot \frac{1}{\sum_{i=1}^{10}x_i} = \frac{1}{3}$ \\
$\displaystyle p_i = 0\cdot \frac{1}{\sum_{i=1}^{10}x_i} = 0, \; i=\overline{4, 10}$ \\
$\displaystyle q_1 = q_2 = q_3 = \frac{1}{9}\cdot \frac{1}{\sum_{i=1}^{10}y_i} = \frac{1}{3}$ \\
$\displaystyle q_i = 0\cdot \frac{1}{\sum_{i=1}^{10}y_i} = 0, \; i=\overline{4, 10}$ \\
\\
\\
\textbf{Задача 6}
Составим матрицу выигрышей:
\begin{center}
$H(i, j) =
 \begin{pmatrix}
  2 & -3 & 4 & -5 & 6 \\
  -3 & 4 & -5 & 6 & -7 \\
  4 & -5 & 6 & -7 & 8 \\
  -5 & 6 & -7 & 8 & -9 \\
  6 & -7 & 8 & -9 & 10 \\
 \end{pmatrix}$
\end{center}
$\displaystyle\underline{I} = \max_{i}\min_{j}H(i, j) = -5$ \\
$\displaystyle\overline{I} = \min_{j}\max_{i}H(i, j) = 6 \neq \underline{I}$ \\
Значит, игра неразрешима в чистых стратегиях. Построим пару двойственных задач линейного программирования, предварительно добавив 10 ко всем элементам матрицы, чтобы они стали положительными: \\
$\begin{cases}
\displaystyle\sum_{i=1}^{5}x_i \to \min \\
12x_1 + 7x_2 + 14x_3 + 5x_4 + 16x_5 \geq 1 \\
7x_1 + 14x_2 + 5x_3 + 16x_4 + 3x_5 \geq 1 \\
14x_1 + 5x_2 + 16x_3 + 3x_4 + 18x_5 \geq 1 \\
5x_1 + 16x_2 + 3x_3 + 18x_4 + x_5 \geq 1 \\
16x_1 + 3x_2 + 18x_3 + x_4 + 20x_5 \geq 1 \\

x_i \geq 0\; i=\overline{1, 5}
\end{cases}$\\
$\begin{cases}
\displaystyle\sum_{i=1}^{5}y_i \to \max \\
12y_1 + 7y_2 + 14y_3 + 5y_4 + 16y_5 \leq 1 \\
7y_1 + 14y_2 + 5y_3 + 16y_4 + 3y_5 \leq 1 \\
14y_1 + 5y_2 + 16y_3 + 3y_4 + 18y_5 \leq 1 \\
5y_1 + 16y_2 + 3y_3 + 18y_4 + y_5 \leq 1 \\
16y_1 + 3y_2 + 18y_3 + y_4 + 20y_5 \leq 1 \\

y_i \geq 0\; i=\overline{1, 5}
\end{cases}$\\
Решив полученную пару задач (например, симплекс-методом), найдём оптимальные решения:\\
$x_1 = \frac{1}{80}$ \\
$x_2 = x_3 = 0$\\
$x_4 = \frac{1}{20}$\\
$x_5 = \frac{3}{80}$\\
$y_1 = \frac{1}{80}$ \\
$y_2 = y_3 = 0$\\
$y_4 = \frac{1}{20}$\\
$y_5 = \frac{3}{80}$\\
Значение игры: \\
$\displaystyle I = \frac{1}{\sum_{i=1}^{10}x_i} - 10 = 10 - 10 = 0$\\
Оптимальные смешанные стратегии: \\
$p_1 = \frac{1}{8}$ \\
$p_2 = p_3 = 0$\\
$p_4 = \frac{1}{2}$\\
$p_5 = \frac{3}{8}$\\
$q_1 = \frac{1}{8}$ \\
$q_2 = q_3 = 0$\\
$q_4 = \frac{1}{2}$\\
$q_5 = \frac{3}{8}$\\
\\
\\

\textbf{Задача 8}

Проверить, являются ли данные смешанные стратегии и значение игры:
\begin{equation*}
    p = \begin{pmatrix}\frac{1}{3} & \frac{1}{3} & \frac{1}{3} \end{pmatrix}, 
    q = \begin{pmatrix}\frac{1}{2} & 0 & 0  & \frac{1}{2} \end{pmatrix},
    I = 0.4.
\end{equation*} \par
решением матричной игры с выигрышами:
\begin{equation*}
    H = \begin{pmatrix} 
            0.8 & 0 & 0 & 0 \\
            0.4 & 0.6 & 0.6 & 0.4 \\
            0 & 0 & 0 & 0.8 \\
        \end{pmatrix}
\end{equation*}

\textbf{Решение.} 

Найдём верхние и нижние значения игры по формулe:
$$
    \underline{I} = \overline{I} = \sum_{i=1}^n\sum_{j=1}^mH(i,j) p_i q_j
$$

$$
    \underline{I} = \overline{I} = 0.8 \cdot \frac{1}{3} \cdot \frac{1}{2} + 0.4 \cdot \frac{1}{3} \cdot \frac{1}{2} + 0.6 \cdot \frac{1}{3} \cdot 0 +
$$ $$ +
	0.6 \cdot \frac{1}{3} \cdot 0 + 0.4 \cdot \frac{1}{3} \cdot \frac{1}{2} + 0.8 \cdot \frac{1}{3} \cdot \frac{1}{2} = 0.4
$$

$$
    \underline{I} = \overline{I} = I = 0.4
$$

Значит, стратегии $p$ и $q$ и значение игры $I$ являются решением матричной игры для $H$.
\newline

\textbf{Задача 9}

Проверить, являются ли данные смешанные стратегии и значение игры:
\begin{equation*}
    p = \begin{pmatrix}\frac{1}{4} & 0 & \frac{1}{4} & \frac{1}{2} \end{pmatrix}, 
    q = \begin{pmatrix}\frac{1}{3} & \frac{1}{3} & \frac{1}{3} \end{pmatrix},
    I = 4.
\end{equation*} \par
решением матричной игры с выигрышами:
\begin{equation*}
    H = \begin{pmatrix} 
            14 & -4 & 2 \\
            -4 & 8 & 8 \\
            4 & 4 & 4 \\
            2& 8 & 2
        \end{pmatrix}
\end{equation*}

\textbf{Решение.} 

Найдём верхние и нижние значения игры по формулe:
$$
    \underline{I} = \overline{I} = \sum_{i=1}^n\sum_{j=1}^mH(i,j) p_i q_j
$$

$$
    \underline{I} = \overline{I} = 14 \cdot \frac{1}{4} \cdot \frac{1}{3} - 4 \cdot \frac{1}{4} \cdot \frac{1}{3} + 2 \cdot \frac{1}{4} \cdot \frac{1}{3} -
$$ $$
    - 4 \cdot 0 \cdot \frac{1}{3} + 8 \cdot 0 \cdot \frac{1}{3} + 8 \cdot 0 \cdot \frac{1}{3} + 
$$ $$
    + 4 \cdot \frac{1}{4} \cdot \frac{1}{3} + 4 \cdot \frac{1}{4} \cdot \frac{1}{3} + 4 \cdot \frac{1}{4} \cdot \frac{1}{3} +
$$ $$
	+ 2 \cdot \frac{1}{2} \cdot \frac{1}{3} + 8 \cdot \frac{1}{2} \cdot \frac{1}{3} + 2 \cdot \frac{1}{2} \cdot \frac{1}{3} = 4
$$

$$
    \underline{I} = \overline{I} = I
$$

Поэтому стратегии $p$ и $q$ и значение игры $I$ являются решением матричной игры для $H$.
\newline

\end{document}

